\documentclass[8pt]{beamer}

% unverzichtbare Mathe-Befehle
\usepackage{amsmath}
% viele Mathe-Symbole
\usepackage{amssymb}
% Erweiterungen für amsmath
\usepackage{mathtools}

% Fonteinstellungen
\usepackage{fontspec}
% Latin Modern Fonts werden automatisch geladen

% deutsche Spracheinstellungen
\usepackage[ngerman]{babel}
%BUG in Biblatex wird hiermit gefixt
\providetoggle{blx@lang@captions@english}

\usepackage[
  math-style=ISO,    % ┐
  bold-style=ISO,    % │
  sans-style=italic, % │ ISO-Standard folgen
  nabla=upright,     % │
  partial=upright,   % ┘
  warnings-off={           % ┐
    mathtools-colon,       % │ unnötige Warnungen ausschalten
    mathtools-overbracket, % │
  },                       % ┘
]{unicode-math}

% traditionelle Fonts für Mathematik
\setmathfont{Latin Modern Math}
\setmathfont{XITS Math}[range={scr, bfscr}]
\setmathfont{XITS Math}[range={cal, bfcal}, StylisticSet=1]

% Zahlen und Einheiten
\usepackage[
  locale=DE,                   % deutsche Einstellungen
  separate-uncertainty=true,   % immer Fehler mit \pm
  per-mode=symbol-or-fraction, % / in inline math, fraction in display math
]{siunitx}

% richtige Anführungszeichen
\usepackage[autostyle]{csquotes}

% schöne Brüche im Text
\usepackage{xfrac}

% Grafiken können eingebunden werden
\usepackage{graphicx}

% Grafiken können in LaTex gemalt werden
\usepackage{tikz, pgfplots}

% Ermöglicht relative Positionierung von tikz-Nodes
\usetikzlibrary{positioning}

% Für Feynman-Graphen mit Tikz
\usepackage{feynmp-auto}

% Für komplexere Captions
\usepackage{caption}

% Literaturverzeichnis
\usepackage[
  backend=biber,
  style=authoryear,
  autocite=inline,
]{biblatex}
% Quellendatenbank
\addbibresource{presentation.bib}

\usetheme[numbering=fraction]{metropolis}

% Für die Titelseite
%\title{The Influence of Music Genres on Performance Metrics: A Comparative Analysis of Spotify and YouTube}
\title{The Genre Factor: Unveiling the Influence of Music Genres on Spotify and YouTube Performance}
\subtitle{Project Presentation - ML Seminar 2023}
\author{Henry Krämerkämper\\%
  \and%
  Christopher Breitfeld}
\institute{Technische Universität Dortmund}
\date{25.05.2023}
\logo{\includegraphics[height=0.5cm]{figures/tu_logo_sw_klein.pdf}}

\begin{document}

\begin{frame}
  \titlepage
\end{frame}

\begin{frame}{Definition and motivation of the problem/task}
  \begin{alertblock}{Definition}
    \begin{itemize}
      \item Does the genre have an influence on a songs performance?
      \item Does the influence differ between YouTube and Spotify?
    \end{itemize}
  \end{alertblock}
  \begin{alertblock}{Motivation}
    \begin{block}{Maximizing song performance:}
      \begin{itemize}
        \item{Understanding the impact of music genres on streaming platforms can help artists, labels, and producers maximize songs performance.}
      \end{itemize}
    \end{block}
    \begin{block}{User behavior and genre analysis:}
      \begin{itemize}
       \item{Analyzing the relationship between genres and user behavior can provide valuable insights for music enthusiasts and researchers.}
      \end{itemize}
    \end{block}
  \end{alertblock}
\end{frame}

\begin{frame}{Description of the Data Set}
  \begin{alertblock}{Dataset from \href{https://www.kaggle.com/datasets/salvatorerastelli/spotify-and-youtube}{Kaggle}: Spotify and YouTube}
	\begin{itemize}
      \item Contains statistics of songs on Spotify and YouTube 
      \item including streams on Spotify and number of views on YouTube
      \item 20.7k entries by 2k artists
      \item Does \alert{NOT} include the genre
      \item Licensed under CC0: Public Domain
    \end{itemize}
  \end{alertblock}
  \begin{alertblock}{Wikipedia Query for the Top-Genre of the Artist}
	\begin{itemize}
     \item Query artist's Wikipedia page for genre names
     \item Choose a list of broader genres, so that the genres are not too specific
     \item Match artist to one of the genres on the list, based on the query
     \item Usage possible under CC-by-SA-3.0
    \end{itemize}
  \end{alertblock}
  \begin{alertblock}{Target of the resulting Dataset}
	\begin{itemize}
     \item Target: genre
     \item Working hypothesis: train neural network to classify
     \item No previous work possible with this dataset due to missing genre
    \end{itemize}
  \end{alertblock}
\end{frame}

\begin{frame}{Strategy for the Comparison with alternative Methods}
  \begin{alertblock}{Motivation for Machine Learning}
	\begin{itemize}
%          \item ML method needed; complex problem without clear definitions for the different classes (genres)
          \item ML needed for complex problem with unclear genre definitions
          \item Genre classification challenging even for humans
    \end{itemize}
  \end{alertblock}
  \begin{alertblock}{Alternative Method: kNN}
    \begin{itemize}
          \item Comparison with kNN-approach
          \item Simplest solution for classification tasks
          \item kNN may be slow for large datasets; expect speed and performance improvements with our method
    \end{itemize}
  \end{alertblock}
\end{frame}


% Alle Quellen, die bisher nicht zitiert wurden
%\nocite{Theories_of_grbs}

\printbibliography

\appendix
\begin{frame}{Appendix: The features of our dataset}
  \begin{alertblock}{Features}
    \begin{columns}[t]
      \begin{column}{0.5\textwidth}
        \begin{itemize}
          \item Track
          \item Artist
          \item Url\_Spotify 
          \item Album 
          \item Album\_type 
          \item Uri 
          \item Danceability
          \item Energy 
          \item Key 
          \item Loudness 
          \item Speechiness 
          \item Acousticness 
          \item Instrumentalness
          \item Liveness 
          \item Tempo 
        \end{itemize}
      \end{column}
      \begin{column}{0.5\textwidth}
        \begin{itemize}
          \item Duration\_ms 
          \item Stream 
          \item Url\_youtube 
          \item Title 
          \item Channel
          \item Views 
          \item Likes 
          \item Comments 
          \item Description 
          \item Licensed
          \item official\_video
        \end{itemize}
      \end{column}
    \end{columns}
  \end{alertblock}
\end{frame}

% \begin{frame}{Appendix: Das Synchrotron Burnoff Limit}

% \end{frame}

% \begin{frame}{Appendix: Detektion eines Gammablitzes: Methoden}

% \end{frame}

% \begin{frame}{Appendix: Detektion eines Gammablitzes: Schwierigkeiten}

% \end{frame}

\end{document}

